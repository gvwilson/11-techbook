\documentclass[10pt,letterpaper]{article}

\input{settings}

\begin{document}
\vspace*{0.2in}

\begin{flushleft}
{\Large
\textbf\newline{Eleven Quick Tips for Writing a Technical Book}
}
\newline
\\
{Greg Wilson}\textsuperscript{1*}
\\
\bigskip
\textbf{1} Third Bit, Toronto, Ontario, Canada\\
* gvwilson@third-bit.com. \\
\end{flushleft}

\section*{Introduction}

I have written four books on programming (with one more the works),
co-authored three others,
and edited six.
Here's what I've learned along the way.

\section*{Tip 1: Don't write for money.}

Your publisher will give you anything from 12\% to 50\% of revenue in royalties.
Assuming a US\$50 price tag and sales of 200–1000 copies,
you will earn between US\$1200 and US\$25,000.
The former is much more likely than the latter;
since it's going to take at least a thousand hours to write the book,
you'll almost certainly be better off doing some consulting
or possibly even getting a part-time job in a fast food restaurant.

So why both writing?

\begin{description}

\item[To build your reputation.]
  Several large technical publishers lose money on the books,
  but make it back because they can charge more for the classes and workshops you teach
  because you're famous.

\item[To give back to the community.]
  None of us got here without help,
  and I think we have a moral obligation to repay that by helping others.

\item[Because you enjoy doing it.]
  Some people like working with wood or wool;
  you might like working with words.

\end{description}

\section*{Tip 2: Don't start with a book.}

Writing a book is a big undertaking, and it's easy to be discouraged.
Instead of tackling it head-on,
start blogging regularly:
one article a week, each a few hundred to a thousand words long.
Don't try to write them in the order you think they'll eventually go into the book—one reason to blog
is to get a better idea of what should come before what—but
as you become more comfortable,
try writing multi-post series that can eventually become chapters.

\section*{Tip 3: Read, then write.}

Go through two or three of your favorite technical books
and make notes about what their authors have done well.
Doing this will give you some ideas for your own book,
but it will also help you become more conscious of your writing—in musical terms,
it will help you learn how to listen to yourself while you're playing.
I learned a lot from the class Unix books by Kernighan et al
\cite{Ke1979, Ke1981, Ke1983, Ke1988} and from \cite{Ud1999},
but your tastes may vary.

\section*{Tip 4: Start with a learner persona.}

A learner persona is a short description of who you're trying to teach,
what they already know,
and what \emph{they} want to learn \cite{Wi2019}.
Create one or two that are fairly similar and write for them,
because a book that's meant for everyone is actually useful to no-one.
And keep in mind the difference between a novice (who is trying to build a mental model),
a competent practitioner (who has one and wants to fill in gaps in their knowledge),
and an expert (who wants higher-level discussion of tradeoffs and alternatives):
no single book can serve all three well.

As a corollary,
\textbf{don't write a general introduction to a topic}
unless you're going to be one of the first two or three to market.
Instead,
focus on a particular aspect (e.g., ``Python for Web Scraping'')
or a particular audience (e.g., ``R for Librarians'').
Your potential audience may be smaller,
but you'll reach a much higher percentage of that audience,
and a focused book is much easier to write.

\section*{Tip 5: Avoid common mistakes.}

\begin{description}

\item[Avoid banal advice]
  (like "avoid common mistakes").
  Few things are as frustrating to the reader as sentences like,
  "You should carefully consider users' needs,"
  because no sensible person would recommend the opposite.
  If you can't provide a checklist of specific things to consider,
  or a scenario to give the reader insight into how you think through that class of problem,
  you're giving them the intellectual equivalent of junk calories.

\item[Don't write to cover your ass.]
  I've read or reviewed dozens of books that start with a short introduction to XYZ.
  I cannot remember a single one that was actually useful:
  if your audience knows Python,
  they don't need a chapter-length intro to the language,
  and if they don't,
  one isn't going to help them.
  I eventually realized that authors included those short introductions
  so that they could claim the book was suitable for novices
  when they knew in their hearts it wasn't.

\item[Don't try to be funny.]
  Very few jokes are funny the second time you hear them;
  even fewer can stand a third re-telling unless they're very, very dry.
  And on a related note,
  please don't use exclamation marks:
  what you're writing probably isn't surprising,
  and certainly won't be the second time around\footnote{
    And remember: nobody reads footnotes.
  }.

\item[Don't try to compete with the internet.]
  Reference manuals were invaluable to me forty years ago,
  but serve little purpose when the world's knowledge is just a click away.
  You add value by explaining the \emph{how} rather than recapitulating the \emph{what},
  so do that.

\end{description}

\section*{Tip 6: Try it out in pieces.}

If I had run a workshop based on the material for my first book
while I was writing it,
I would have realized that several of my core assumptions about my audience were completely wrong
while I still had time and energy to fix things.
Conferences are always looking for tutorials,
your colleagues will probably welcome some lunch and learn sessions,
and if all else fails you can create a YouTube channel
to find out how your material sounds.

A corollary is that
there's no point trying material out if you don't \textbf{incorporate what you learn}.
At a guess,
every line of my most recent book has been rewritten at least twice,
and a quick glance in the version control log tells me that
some parts have been revisited over a dozen times.

\begin{mdframed}

  No matter how carefully you plan,
  you will hit part 5 and realize that
  there was something you shoulve have explained back in part 3.
  You can fill this in on the fly if you are teaching in person,
  but that doesn't work in a book.
  For this reason,
  a live lesson should be the \emph{intersection} of what audience members don't know
  (i.e., the things you're sure they all need)
  while a book has to be the \emph{union} of what readers don't know
  (i.e., the things any of them might need).

\end{mdframed}

\section*{Tip 7: Drywall, then paint.}

In other words write a rough draft with placeholders
and \emph{then} worry about diagrams, bibliography citations, glossary entries, and so on.
The word \texttt{FIXME} appeared over a hundred times in first draft of \cite{Wi2022};
I changed my mind about more than half of those (or simply discarded them)
by the time the text settled down.

An important special case of this is that you should do diagrams on a whiteboard first.
It's faster and more flexible than any computer drawing tool
(even fingertip sketching apps for tablets)
and a photo taken with a cell phone is good enough for your first readers.

\section*{Tip 8: Automate, but proofread.}

This is the technical author's equivalent of ``trust, but verify''.
With a single command I can re-run all of the examples in my latest book,
capture their output
and insert it in the manuscript,
but I still have to re-read the discussion about those examples every time I make a change
to make sure they haven't fallen out of step.
Similarly,
it doesn't matter how well you know your chosen subject:
something will have changed since the last time you looked,
and something else will change in the year or more it takes you to complete your manuscript.

\begin{mdframed}

I have worked on books using LaTeX, Microsoft Word, computational notebook,
and Markdown-based static site generators,
and they are all frustrating.
Tools that store everything as lines of text with manually-typed formatting commands
and a compilation step for previewing
impose a high cognitive load,
particularly when it comes to creating tables or diagrams,
while WYSIWYG tools don't play nicely with version control.
There is no good technical reason for us to have to choose,
but until programmers stop insisting on backward compatibility with punch cards,
authors have to accept frustration as a fact of life.

\end{mdframed}

\section*{Tip 9: Trust your reviewers and your editor.}

Lots of people will do cursory reviews
or give your manuscript a single careful read;
people who will look at changes over and over again
and see what's actually on the page each time
are worth their weight in rubies.
Trust them:
if a passage doesn't make sense to them,
then it doesn't matter if it makes sense to you.

Similarly,
as one of my editors once said,
an author's job is to produce the manure
in which an editor grows something worth reading.
You may sweat over an example for days,
but your audience doesn't see your effort:
they see your results.
If explanation that seem clear, concise, and elegant to you don't make sense to your readers,
then they're right and you're wrong.

\section*{Tip 10: Don't be afraid to set it aside.}

I wrote an introduction to R for Python programmers that will probably never see the light of day,
and have stopped work on an undergraduate guide to working on team projects.
In both cases I found that the more I wrote,
the less I believed the book would actually help its intended audience.
I hope I'll get back to the second book eventually,
but I won't feel ashamed if I don't:
scientists don't expect every hypothesis to turn out to be true,
and authors shouldn't either.

\section*{Tip 11: Stop, ship, and celebrate.}

Your book will never be perfect.
It will probably never even feel finished,
any more than software does,
but you should ship it anyway
and then celebrate what you've accomplished.

\bibliographystyle{unsrt}
\bibliography{11-techbook}

\end{document}
